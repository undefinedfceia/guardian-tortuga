\begin{lstlisting}[language=C++]
#define MAXP 100000	//no necesariamente primo

int criba[MAXP+1];

void crearcriba(){
	int w[] = {4,2,4,2,4,6,2,6};
	for(int p=25;p<=MAXP;p+=10) criba[p]=5;
	for(int p=9;p<=MAXP;p+=6) criba[p]=3; 
	for(int p=4;p<=MAXP;p+=2) criba[p]=2;
	for(int p=7,cur=0;p*p<=MAXP;p+=w[cur++&7]) if (!criba[p]) 
		for(int j=p*p;j<=MAXP;j+=(p<<1)) if(!criba[j]) criba[j]=p;
}

vector<int> primos;
void buscarprimos(){
	crearcriba();
	forr (i,2,MAXP+1) if (!criba[i]) primos.push_back(i);
}

//factoriza bien numeros hasta MAXP^2
map<ll,ll> fact(ll n){ //O (cant primos)
	map<ll,ll> ret;
	forall(p, primos){
		while(!(n%*p)){
			ret[*p]++;//divisor found
			n/=*p;
		}
	}
	if(n>1) ret[n]++;
	return ret;
}

//factoriza bien numeros hasta MAXP
map<ll,ll> fact2(ll n){ //O (lg n)
	map<ll,ll> ret;
	while (criba[n]){
		ret[criba[n]]++;
		n/=criba[n];
	}
	if(n>1) ret[n]++;
	return ret;
}

int main() {
	buscarprimos();
	return 0;
}
\end{lstlisting}
